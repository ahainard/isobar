\newcommand{\minitab}[2][l]{\begin{tabular}{#1}#2\end{tabular}}
\newcommand{\minipagec}[2][l]{\begin{minipage}{#1}#2\end{minipage}}
\newcommand{\boxplot}[7]{
  \begin{tikzpicture}[x=2cm,line width=0.5pt,node distance=0.1]
      \tikzboxplot{#1}{#2}{#3}{#4}{#5}{#6}{#7}
  \end{tikzpicture}
}


\newcommand{\isobarthanks}{\thanks{This report was 
  generated using the \texttt{isobar} R package. If you use it in published work, 
  please cite 'Breitwieser FP et al.  : General statistical 
  modeling of data from protein relative expression isobaric tags, 
  \textit{Journal of Proteome Research} 2011'}}

\newcommand{\uniprotlink}[1]{\href{http://www.uniprot.org/uniprot/#1}{#1}}
\tikzset{
  pentagon/.style = {
    regular polygon,regular polygon sides=5
  },
  rectangle/.style = {
    regular polygon,regular polygon sides=4
  },
  rs/.style={
    thick,circle,draw,draw=black!80,fill=black!20
  },
  gs/.style={
    thick,rectangle,draw,draw=black!50,fill=black!20
  },
  quant peptide/.style={
    thick,rectangle,draw,draw=#1!50,fill=#1!20
  },
  quant peptide/.default=red,
  us/.style={
    thick,circle,draw,draw=black!30,fill=black!10
  },
  quant peptide red/.style={
    thick,rectangle,draw,draw=red!50,fill=red!20
  },
  quant peptide green/.style={
    thick,rectangle,draw,draw=green!50,fill=green!20
  },
  quant peptide blue/.style={
    thick,rectangle,draw,draw=blue!50,fill=blue!20
  },
  quant peptide cyan/.style={
    thick,rectangle,draw,draw=cyan!50,fill=cyan!20
  },
  quant peptide magenta/.style={
    thick,rectangle,draw,draw=magenta!50,fill=magenta!20
  },
  quant peptide yellow/.style={
    thick,rectangle,draw,draw=yellow!50,fill=yellow!20
  }
%colors <- c("red","green","blue","cyan","magenta","yellow")
}

% information in protein group
%  - rows: proteins
%    - gene name / protein name?
%  - columns: peptides
%    - peptide type: reporter specific, group specific, unspecific

% parameters:
%  #1: protein position
%  #2: protein AC
\newcommand{\proteincol}[5]{
  \node [anchor=east,draw,minimum size=0.2cm] (prot#1) at (-3,#1) {#2};
  \node [minimum size=0.2cm] at (-2,#1) {#3};
  \node [minimum size=0.2cm] at (-1,#1) {#4};
  \node [minimum size=0.2cm] at (0,#1) {#5};
}

% parameters:
%  #1: protein position
%  #2: style / parameters for node
%  #3: peptide numbers
%  #4: node text
\newcommand{\proteinrow}[3]{
    \foreach \x/\style in {#2}{
    \node (prot#1 pep\x) at (\x,#1) [\style] {#3};
  }
}

\renewcommand{\ll}{0.25}
\newcounter{nplot}

\newcommand{\sdboxplot}[6]
{
  \pgfmathsetmacro{\median}{#1}
  \pgfmathsetmacro{\sd}{#2}
  \pgfmathparse{max(-\bnd,\median-\sd)}
  \pgfmathsetmacro{\ql}{\pgfmathresult} % first quartile
  \pgfmathparse{max(-\bnd,\median-2*\sd)}
  \pgfmathsetmacro{\cl}{\pgfmathresult} % lower border of CI
  \pgfmathparse{min(\bnd,\median+\sd)}
  \pgfmathsetmacro{\qh}{\pgfmathresult} % third quartile
  \pgfmathparse{min(\bnd,\median+2*\sd)}
  \pgfmathsetmacro{\ch}{\pgfmathresult} % upper border of CI
  \pgfmathsetmacro{\hh}{0+2*\ll*\value{nplot}}
  \def\divcol{#3}

  \filldraw[fill=#3] (\ql,\hh) rectangle (\qh,\hh+\ll);% draw the box
  \draw (\median,\hh) -- (\median,\hh+\ll) ;% draw the median
  \draw (\qh,\hh+\ll/2) -- (\ch,\hh+\ll/2);% draw right whisker
  \draw (\ql,\hh+\ll/2) -- (\cl,\hh+\ll/2);% draw left whisker
  \draw (\ch,\hh+\ll/2-\ll/4) -- (\ch,\hh+\ll/2+\ll/4);% draw vertical tab
  \draw (\cl,\hh+\ll/2-\ll/4) -- (\cl,\hh+\ll/2+\ll/4);% draw vertical tab
  \draw [color=black!20] (-\bnd,-0.1) -- (\bnd,-0.1); % Axis
  \node[#4] (bdr) at (-\bnd-0.1,\hh+\ll/2) {\tt{<}} ;
  \node[#5] at (\bnd+0.1,\hh+\ll/2) {\tt{>}};
  \node (reagent) [left=of bdr,text=black] {#6};

  \addtocounter{nplot}{1}
}

\newcommand{\tikznode}[2]
{
	\begin{tikzpicture} \node(#1) {#2}; \end{tikzpicture}
}

\newcommand{\noboxplot}[2]
{
  \node (reagent) [text=black] {#1 \textbf{#2}};

  \addtocounter{nplot}{1}
}


\newcommand{\rsdboxplot}[9]
{
  \pgfmathsetmacro{\median}{#1}
  \pgfmathsetmacro{\sd}{#2}
  \pgfmathparse{max(-\bnd,\median-\sd)}
  \pgfmathsetmacro{\ql}{\pgfmathresult} % first quartile
  \pgfmathparse{max(-\bnd,\median-1.96*\sd)}
  \pgfmathsetmacro{\cl}{\pgfmathresult} % lower border of CI
  \pgfmathparse{min(\bnd,\median+\sd)}
  \pgfmathsetmacro{\qh}{\pgfmathresult} % third quartile
  \pgfmathparse{min(\bnd,\median+1.96*\sd)}
  \pgfmathsetmacro{\ch}{\pgfmathresult} % upper border of CI
  \pgfmathsetmacro{\hh}{0+2*\ll*\value{nplot}}
  \def\divcol{#3}

  \draw[#9] [lightgray] (-\bnd,\hh+\ll/2) -- (\bnd,\hh+\ll/2); % Axis
  \foreach \x in {-\bnd,0,\bnd}
    \draw[#9] [color=black!20] (\x,\hh) -- (\x,\hh+\ll); % 

  \filldraw[color=#9,fill=#3] (\ql,\hh) rectangle (\qh,\hh+\ll);% draw the box
  \draw[#9] (\median,\hh) -- (\median,\hh+\ll) ;% draw the median
  \draw[#9] (\qh,\hh+\ll/2) -- (\ch,\hh+\ll/2);% draw right whisker
  \draw[#9] (\ql,\hh+\ll/2) -- (\cl,\hh+\ll/2);% draw left whisker
  \draw[#9] (\ch,\hh+\ll/2-\ll/4) -- (\ch,\hh+\ll/2+\ll/4);% draw vertical tab
  \draw[#9] (\cl,\hh+\ll/2-\ll/4) -- (\cl,\hh+\ll/2+\ll/4);% draw vertical tab
  \node[#4] (bdr) at (-\bnd-0.1,\hh+\ll/2) {\tt{<}} ;
  \node[#5] at (\bnd+0.1,\hh+\ll/2) {\tt{>}};
  \node (reagent) [left=of bdr,text=black] {#6 \textbf{#7}};

  \addtocounter{nplot}{1}
}


\newcommand{\rsdboxplotnoname}[7]
{
  \pgfmathsetmacro{\median}{#1}
  \pgfmathsetmacro{\sd}{#2}
  \pgfmathparse{max(-\bnd,\median-\sd)}
  \pgfmathsetmacro{\ql}{\pgfmathresult} % first quartile 
  \pgfmathparse{max(-\bnd,\median-1.96*\sd)}
  \pgfmathsetmacro{\cl}{\pgfmathresult} % lower border of CI
  \pgfmathparse{min(\bnd,\median+\sd)}
  \pgfmathsetmacro{\qh}{\pgfmathresult} % third quartile
  \pgfmathparse{min(\bnd,\median+1.96*\sd)}
  \pgfmathsetmacro{\ch}{\pgfmathresult} % upper border of CI
  \pgfmathsetmacro{\hh}{0+2*\ll*\value{nplot}}
  \def\divcol{#3}
  \draw[#7] [lightgray] (-\bnd,\hh+\ll/2) -- (\bnd,\hh+\ll/2); % Axis
  \foreach \x in {-\bnd,0,\bnd}
    \draw[#7] [color=black!20] (\x,\hh) -- (\x,\hh+\ll); %                                                                    

  \filldraw[color=#7,fill=#3] (\ql,\hh) rectangle (\qh,\hh+\ll);% draw the box
  \draw[#7] (\median,\hh) -- (\median,\hh+\ll) ;% draw the median
  \draw[#7] (\qh,\hh+\ll/2) -- (\ch,\hh+\ll/2);% draw right whisker
  \draw[#7] (\ql,\hh+\ll/2) -- (\cl,\hh+\ll/2);% draw left whisker
  \draw[#7] (\ch,\hh+\ll/2-\ll/4) -- (\ch,\hh+\ll/2+\ll/4);% draw vertical tab
  \draw[#7] (\cl,\hh+\ll/2-\ll/4) -- (\cl,\hh+\ll/2+\ll/4);% draw vertical tab
  \node[#4] (bdr) at (-\bnd-0.1,\hh+\ll/2) {\tt{<}} ;
  \node[#5] at (\bnd+0.1,\hh+\ll/2) {\tt{>}};

  \addtocounter{nplot}{1}
}

\newcommand{\pgfbndset}[2]
{
%  \pgfmathparse{min(\bnd,max(-\bnd,#1))}
  \pgfmathparse{max(-\bnd,#2)}
  \pgfmathparse{min(\bnd,\pgfmathresult)}
  \pgfmathsetmacro{#1}{\pgfmathresult}
}

\newcommand{\tikzboxplot}[7]
{
  \pgfmathsetmacro{\median}{#1}
  \pgfmathsetmacro{\sd}{#2}
  \pgfbndset{\ql}{\median-\sd}      % first quartile 
  \pgfbndset{\cl}{\median-1.96*\sd} % lower border of CI
  \pgfbndset{\qh}{\median+\sd}      % third quartile 
  \pgfbndset{\ch}{\median+1.96*\sd} % upper border of CI
  \pgfbndset{\median}{\median}

  \pgfmathsetmacro{\hh}{0+2*\ll}
  \def\divcol{#3}
  \draw[#7] [lightgray] (-\bnd,\hh+\ll/2) -- (\bnd,\hh+\ll/2); % Axis
  \foreach \x in {-\bnd,0,\bnd}
    \draw[#7] [color=black!20] (\x,\hh) -- (\x,\hh+\ll); %                                                                    
  \filldraw[color=#7,fill=#3] (\ql,\hh) rectangle (\qh,\hh+\ll);% draw the box
  \draw[#7] (\median,\hh) -- (\median,\hh+\ll) ;% draw the median
  \draw[#7] (\qh,\hh+\ll/2) -- (\ch,\hh+\ll/2);% draw right whisker
  \draw[#7] (\ql,\hh+\ll/2) -- (\cl,\hh+\ll/2);% draw left whisker
  \draw[#7] (\ch,\hh+\ll/2-\ll/4) -- (\ch,\hh+\ll/2+\ll/4);% draw vertical tab
  \draw[#7] (\cl,\hh+\ll/2-\ll/4) -- (\cl,\hh+\ll/2+\ll/4);% draw vertical tab
  \node[#4] (bdr) at (-\bnd-0.1,\hh+\ll/2) {\tt{<}} ;
  \node[#5] at (\bnd+0.1,\hh+\ll/2) {\tt{>}};
}


\newcommand{\rsdboxplotold}[9]
{
  \pgfmathsetmacro{\median}{#1}
  \pgfmathsetmacro{\sd}{#2}
  \pgfmathparse{max(-\bnd,\median-\sd)}
  \pgfmathsetmacro{\ql}{\pgfmathresult} % first quartile
  \pgfmathparse{max(-\bnd,\median-1.96*\sd)}
  \pgfmathsetmacro{\cl}{\pgfmathresult} % lower border of CI
  \pgfmathparse{min(\bnd,\median+\sd)}
  \pgfmathsetmacro{\qh}{\pgfmathresult} % third quartile
  \pgfmathparse{min(\bnd,\median+1.96*\sd)}
  \pgfmathsetmacro{\ch}{\pgfmathresult} % upper border of CI
  \pgfmathsetmacro{\hh}{0+2*\ll*\value{nplot}}
  \def\divcol{#3}

  \filldraw[color=#9,fill=#3] (\ql,\hh) rectangle (\qh,\hh+\ll);% draw the box
  \draw[#9] (\median,\hh) -- (\median,\hh+\ll) ;% draw the median
  \draw[#9] (\qh,\hh+\ll/2) -- (\ch,\hh+\ll/2);% draw right whisker
  \draw[#9] (\ql,\hh+\ll/2) -- (\cl,\hh+\ll/2);% draw left whisker
  \draw[#9] (\ch,\hh+\ll/2-\ll/4) -- (\ch,\hh+\ll/2+\ll/4);% draw vertical tab
  \draw[#9] (\cl,\hh+\ll/2-\ll/4) -- (\cl,\hh+\ll/2+\ll/4);% draw vertical tab
  \draw[#9] [color=black!20] (0,-0.1) -- (0,0); % 
  \draw[#9] [color=black!20] (-\bnd,-0.1) -- (-\bnd,0); % 
  \draw[#9] [color=black!20] (\bnd,-0.1) -- (\bnd,0); % 
  \draw[#9] [color=black!20] (-\bnd,-0.1) -- (\bnd,-0.1); % Axis
  \node[#4] (bdr) at (-\bnd-0.1,\hh+\ll/2) {\tt{<}} ;
  \node[#5] at (\bnd+0.1,\hh+\ll/2) {\tt{>}};
  \node (reagent) [left=of bdr,text=black] {#6 \textbf{#7}};

  \addtocounter{nplot}{1}
}

\newcommand{\drawaxis}[2]{
\scriptsize
\begin{tikzpicture}[ x=2cm,font=\sffamily,color=black!60]
  \draw (-\bnd,0) -- coordinate (x axis mid) (\bnd,0);
  \draw (-\bnd,0pt) -- (-\bnd,#1) node[anchor=#2] {-\bndt};
  \draw (0,0pt) -- (0,#1) node[anchor=#2] {1};
  \draw (\bnd,0pt) -- (\bnd,#1) node[anchor=#2] {\bndt};
  \node[black!1] (bdr) at (-\bnd-0.1,\ll/2) {\tt{ }} ;
  \node[black!1] at (\bnd+0.1,\ll/2) {\tt{ }};
\end{tikzpicture}
\normalsize
}

\newcommand*\up{\textcolor{green}{%
  \ensuremath{\blacktriangle}}}
\newcommand*\down{\textcolor{red}{%
  \ensuremath{\blacktriangledown}}}
\newcommand*\const{\textcolor{darkgray}%
  {\textbf{--}}}

\newcommand{\drawbrace}[2]{
\draw[snake=brace] (#1.south east) -- (#1.south west) node [pos=0.5,below] {\small #2};
}

\newcommand{\oldboxplot}[9]
{
  \pgfmathsetmacro{\median}{#1}
  \pgfmathsetmacro{\sd}{#2}
  \pgfmathsetmacro{\ql}{#3} % first quartile
  \pgfmathsetmacro{\cl}{#4} % lower border of CI
  \pgfmathsetmacro{\qh}{#5} % third quartile
  \pgfmathsetmacro{\ch}{#6} % upper border of CI
  \newcommand{\divcol}{#7}

  \begin{tikzpicture}[x=3cm,color=black!60,line width=0.5pt]
  \filldraw[fill=\divcol] (\ql,0) rectangle (\qh,\ll);% draw the box
  \draw (\median,0) -- (\median,\ll) ;% draw the median
  \draw (\qh,\ll/2) -- (\ch,\ll/2);% draw right whisker
  \draw (\ql,\ll/2) -- (\cl,\ll/2);% draw left whisker
  \draw (\ch,\ll/2-\ll/4) -- (\ch,\ll/2+\ll/4);% draw vertical tab
  \draw (\cl,\ll/2-\ll/4) -- (\cl,\ll/2+\ll/4);% draw vertical tab
  \draw [color=black!20] (-\bnd,-0.1) -- (\bnd,-0.1); % Axis
  \fill[fill=#8] (-\bnd-0.1,0) rectangle (-\bnd-0.2,\ll);
  \fill[fill=#9] (\bnd+0.1,0) rectangle (\bnd+0.2,\ll);
  \end{tikzpicture}
}
