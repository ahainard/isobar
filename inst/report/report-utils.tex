\newcommand{\minitab}[2][l]{\begin{tabular}{#1}#2\end{tabular}}
\newcommand{\minipagec}[2][l]{\begin{minipage}{#1}#2\end{minipage}}
\newcommand{\uniprotlink}[1]{\href{http://www.uniprot.org/uniprot/#1}{#1}}

\newcommand{\rr}{\raggedright}
\newcommand{\tn}{\tabularnewline}
\newcommand{\rh}[1]{\begin{rotate}{65}\textbf{#1}\end{rotate}}
\newcommand{\rth}[1]{\begin{turn}{65}\textbf{#1}\end{turn}}
\newcommand{\secsubtitle}[1]{\textit{#1}}

\newcolumntype{f}{>{\footnotesize}r} 

\newcommand{\boxplot}[7]{
  \begin{tikzpicture}[x=2cm,line width=0.5pt,node distance=0.1,baseline=3.8ex]
      \tikzboxplot{#1}{#2}{#3}{#4}{#5}{#6}{#7}
  \end{tikzpicture}
}

% parameters:
%  #1: protein position
%  #2: protein AC
\newcommand{\proteincol}[5]{
  \node [anchor=east,draw,minimum size=0.2cm] (prot#1) at (-3,#1) {#2};
  \node [minimum size=0.2cm] at (-2,#1) {#3};
  \node [minimum size=0.2cm] at (-1,#1) {#4};
  \node [minimum size=0.2cm] at (0,#1) {#5};
}

% parameters:
%  #1: protein position
%  #2: style / parameters for node
%  #3: peptide numbers
%  #4: node text
\newcommand{\proteinrow}[3]{
    \foreach \x/\style in {#2}{
    \node (prot#1 pep\x) at (\x+1,#1*0.75+0.75) [\style] {#3};
  }
}

\renewcommand{\ll}{0.25}
\newcounter{nplot}


\tikzset{
  rect/.style = {
    regular polygon,regular polygon sides=4
  },
  rs/.style={ thick,circle,draw,draw=black!80,fill=black!20 },
  gs/.style={ thick,rect,draw,draw=#1!50,fill=#1!20 },
  us/.style={ thick,circle,draw,draw=black!30,fill=black!10 },
  gs/.default=black,
  gs blue/.style=   { thick,rect,draw,draw=blue!50,fill=blue!20 },
  gs red/.style=    { thick,rect,draw,draw=red!50,fill=red!20 },
  gs green/.style=  { thick,rect,draw,draw=green!50,fill=green!20 },
  gs cyan/.style=   { thick,rect,draw,draw=cyan!50,fill=cyan!20 },
  gs magenta/.style={ thick,rect,draw,draw=magenta!50,fill=magenta!20 },
  gs yellow/.style= { thick,rect,draw,draw=yellow!50,fill=yellow!20 }
}

\newcommand{\pgfbndset}[2]
{
  \pgfmathparse{max(-\bnd,#2)}
  \pgfmathparse{min(\bnd,\pgfmathresult)}
  \pgfmathsetmacro{#1}{\pgfmathresult}
}

\newcommand{\tikzboxplot}[7]
{
  \pgfmathsetmacro{\median}{#1}
  \pgfmathsetmacro{\sd}{#2}
  \pgfbndset{\ql}{\median-\sd}      % first quartile 
  \pgfbndset{\cl}{\median-1.96*\sd} % lower border of CI
  \pgfbndset{\qh}{\median+\sd}      % third quartile 
  \pgfbndset{\ch}{\median+1.96*\sd} % upper border of CI
  \pgfbndset{\median}{\median}

  \pgfmathsetmacro{\hh}{0+2*\ll}
  \def\divcol{#3}
  \draw[#7] [lightgray] (-\bnd,\hh+\ll/2) -- (\bnd,\hh+\ll/2); % Axis
  \foreach \x in {-\bnd,0,\bnd}
    \draw[#7] [color=black!20] (\x,\hh) -- (\x,\hh+\ll); %                                                                    
  \filldraw[color=#7,fill=#3] (\ql,\hh) rectangle (\qh,\hh+\ll);% draw the box
  \draw[#7] (\median,\hh) -- (\median,\hh+\ll) ;% draw the median
  \draw[#7] (\qh,\hh+\ll/2) -- (\ch,\hh+\ll/2);% draw right whisker
  \draw[#7] (\ql,\hh+\ll/2) -- (\cl,\hh+\ll/2);% draw left whisker
  \draw[#7] (\ch,\hh+\ll/2-\ll/4) -- (\ch,\hh+\ll/2+\ll/4);% draw vertical tab
  \draw[#7] (\cl,\hh+\ll/2-\ll/4) -- (\cl,\hh+\ll/2+\ll/4);% draw vertical tab
  \node[#4] (bdr) at (-\bnd-0.1,\hh+\ll/2) {\tt{<}} ;
  \node[#5] at (\bnd+0.1,\hh+\ll/2) {\tt{>}};
}

%% draw boxplot axis
\newcommand{\drawaxis}[2]{
\scriptsize
\begin{tikzpicture}[ x=2cm,font=\sffamily,color=black!60]
  \draw (-\bnd,0) -- coordinate (x axis mid) (\bnd,0);
  \draw (-\bnd,0pt) -- (-\bnd,#1) node[anchor=#2] {-\bndt};
  \draw (0,0pt) -- (0,#1) node[anchor=#2] {1};
  \draw (\bnd,0pt) -- (\bnd,#1) node[anchor=#2] {\bndt};
  \node[black!1] (bdr) at (-\bnd-0.1,\ll/2) {\tt{ }} ;
  \node[black!1] at (\bnd+0.1,\ll/2) {\tt{ }};
\end{tikzpicture}
\normalsize
}

